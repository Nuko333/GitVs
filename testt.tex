\documentclass{ltjsarticle}
% ltjsarticle: lualatex 用の 日本語 documentclass
% 他のタイプセットエンジンを使ってビルドする場合は、 \documentclass[dvipdfmx]{jsarticle} などとする。
\usepackage{url}

\begin{document}
\parindent = 0pt
\begin{center}
{\LARGE 読む文献の一覧} 
\end{center}
\begin{flushright}
\today
\end{flushright}
\section{微積分&線形代数}
黒田、「微分・積分」\\
齋藤毅、「線形代数の世界」\\
笠原、「微分積分学」
\section{複素解析}
神保、「複素関数入門」\\
堀川、「複素関数の要諦」
\section{位相、多様体}
松坂、「集合・位相入門」\\
松本、「多様体の基礎」
\section{微分幾何}
佐古彰史、「ゲージ理論・一般相対性理論のための微分幾何入門」\\
中内伸光、「じっくり学ぶ曲線と曲面―微分幾何学初歩」
\section{微分方程式}
笠原、「微分方程式の基礎」\\
ポントリャーギン、「常微分方程式」
\section{代数幾何、代数位相幾何}
宮西、「代数幾何」\\
桂、「代数幾何入門」\\
加藤、「位相幾何」\\
堀田、「代数入門」\\
星、「群論序説」\\
寺杣、「リーマン面の理論」\\
今野、「リーマン面と代数曲線」\\
アティマク\\
ハーツホーン、シルバーマン\\
Torsten wedhornの本\\
新妻、「可換環論の様相」\\
松村、「可換環論」
\section{ホモロジー代数、圏論}
志甫、ホモロジー代数\\
Goldblatt、『Topoi』
\section{院試}
大学院への演習シリーズ(代数、幾何、解析)\\
梶原、「新修解析学」\\
過去問を繰り返し解く☆☆
\section{webの文献}
Takatani Note\\
\url{https://takataninote.com/#%E5%BE%AE%E5%88%86%E7%A9%8D%E5%88%86}
\end{document}

