\documentclass[11pt]{jsarticle}
%段落の自動字下げをなくす。
\parindent = 0pt

%Page margin
\usepackage[margin=20truemm]{geometry}

%Letters
\usepackage{amsmath,mathtools,amssymb}
\newcommand{\mf}[1]{\mathfrak{#1}} %Shorten mathfrak command
\newcommand{\mb}[1]{\mathbb{#1}} %Shorten mathbb command

%Appearance
    %Page number
    \pagenumbering{arabic}
    \usepackage{lastpage}
    \usepackage{fancyhdr}
    \pagestyle{fancy}
    \fancyhf{} % clear existing header/footer entries
    \fancyfoot[C]{\thepage \hspace{1pt} \slash \hspace{1pt} \pageref*{LastPage}}%Place Page X / Y
    \renewcommand{\headrulewidth}{0pt} %Deleted header line
    \renewcommand{\footrulewidth}{0pt} %Deleted footer line

%設定ファイルの読み込み
\usepackage{statementenv}

%Tikz
\usepackage{tikz}

%囲み枠の用意
\usepackage {framed,xcolor}
%https://atatat.hatenablog.com/entry/cloud_latex14_frame
%https://konoyonohana.blog.fc2.com/blog-entry-242.html
\usepackage {fancybox}
\usepackage{ascmac}

\begin{document}
\thispagestyle{fancy} %これをつけないと\maketitleのせいで右上に1が余分に追加で表示されてしまう.

\begin{shadebox}
   \begin{center}
   \vspace{10pt}
   {\LARGE 円錐について}
   \vspace{10pt}
   \end{center}
\end{shadebox}

\begin{flushright}
\today
\end{flushright}

\section{円錐とは?}
まず、円錐ってどんな図形かを説明するよ。
\begin{statesp}{point}{円錐の説明}
下の図のように、底面が円になっていて、先端がとがっている立体を{\color{red} \bf 円錐}という。底面の円の{\color{red} 半径}$r$で表す。底面から先端までの{\color{red} 高さ}を$h$で表す。また、図で緑色や青色の線で表した、先端と底面を結ぶななめの線を{\color{red} \bf 母線}という。
\end{statesp}
\usetikzlibrary{calc}

\begin{figure}[h!] 
%円錐の書き方は以下を参考!!
%https://tex.stackexchange.com/questions/171169/how-to-draw-a-simple-cone-with-height-and-radius-with-tikz
\centering
\begin{tikzpicture}
    %中心(0,0)で170度から10度まで半径2.4cmの円弧を引く。posは0→1で、170度→0度となる点となる。
    %この場合は、170度のところに(a)を設置
    \draw[dashed] (0,0) arc (170:10:2cm and 0.4cm)coordinate[pos=0](a);
    %中心(0,0)で-170度から-10度まで円弧を引く。pos指定がないため-10度に(b)
    \draw (0,0) arc (-170:-10:2cm and 0.4cm)coordinate (b);
    %底面の左端点(a)と右端点(b)を結ぶのが直径。その中点が底面の中心O
    \fill ($(a)!0.5!(b)$) circle[radius=1pt] coordinate(O);
    %(a)と(b)の中点をy軸方向にシフトした点(頂点)を配置して、その点をCとする。
    \coordinate (c) at ([yshift=4cm]$(a)!0.5!(b)$);
    %posが0→1となるとき、C→(0,0)となる。高さと直角の交わりを(aa)とする。
    %1番目の--は、(a)と(b)の中点OとCを結ぶ線分を表していて、直後のnodeで線分の右側に、文字hを配置した。($(a)!0.5!(b)$)が終点。
    %posが0→1となるとき、O→(b)となる。半径と直角の交わりを(bb)とする。
    %中点Oと(b)を結ぶ線分の上に文字rを配置する。
    \draw[densely dashed,red] (c) -- node[right,red,align=left] {\large 高さ\\$h$}coordinate[pos=0.95] (aa) (O)
                            -- node[above,red] {\large 半径$r$}coordinate[pos=0.1] (bb) (b);
    %(aa)と(bb)を直角に結ぶ
    \draw (aa) -| (bb);
    %頂点Cと底面の左右の端点を結ぶ。つまり母線を引く。
    \draw[very thick,blue] (a) -- node[left]{\large 母線} (c);
    \draw[very thick,green] (c) -- node[right]{\large 母線} (b);
\end{tikzpicture}
\caption{円錐}
\end{figure}

\section{体積、表面積}
よく出てくる問題に、円錐の表面積と体積を求める問題があるよ。どうやって求めるか、整理していこう。
\begin{statesp}[conevol]{formula}{円錐の体積}
高さ$h$、底面の半径$r$の円錐の体積$V$は次のように求める。
\begin{center}
\textcolor{red}{$V$=\ $\frac{1}{3} \times$(底面の円の面積) $\times$ (高さ)} =\ $\frac{1}{3} \times \pi r^2 \times h$ 
\end{center}
まとめると、次のような式になる。
\begin{equation}
\textcolor{red}{V=\frac{1}{3} \pi h r^2}
\end{equation}
\end{statesp}

%\begin{statesp}[ラベル名]{環境名}{ボックスタイトル}となっている。
%\refspの仕方:\refsp{環境名:ラベル名}で参照している。
この\refsp{formula:conevol}は絶対に暗記して、使えるようにしよう。\\
次は、円錐の表面積の簡単な求め方について、見ていこう。

\newpage

\begin{statesp}[conearea]{formula}{円錐の表面積}
高さ$h$、底面の半径$r$、母線の長さ$l$の円錐を考える。円錐の横側の面の面積を\textcolor{red}{側面積}という。下の図$ref$で確かめたほうが分かりやすいと思う。
このとき、円錐の表面積$S$は、次のように求める。
\begin{center}
\textcolor{red}{(側面積)\ =\ (母線) $\times$ (半径)\ =\ $lr$}\\
\textcolor{red}{(表面積$S$)\ =\ (側面積) $+$ (底面積)}\ =\ $lr + \pi r^2$
\end{center}
\end{statesp}

%横にならべるのは、以下を参考に
%https://atatat.hatenablog.com/entry/cloud_latex27_tikz_layou
%扇形の展開図は、以下を参考にした
%https://texwiki.texjp.org/TikZ
\begin{figure}[htbp]
\centering
    \begin{tabular}{cc}
	%%%%第一の図%%%%
      \begin{minipage}[t]{0.35\linewidth}
      \begin{tikzpicture}
    \draw[dashed] (0,0) arc (170:10:2cm and 0.4cm)coordinate[pos=0](a);
    \draw (0,0) arc (-170:-10:2cm and 0.4cm)coordinate (b);
    \fill ($(a)!0.5!(b)$) circle[radius=1pt] coordinate(O);
    \draw[densely dashed] (c) -- node[right,align=left] {高さ\\$h$}coordinate[pos=0.95] (aa) (O);
    \draw[densely dashed,red] (O) -- node[above,red] {\large 半径$r$}coordinate[pos=0.1] (bb) (b);
    %(aa)と(bb)を直角に結ぶ
    \draw (aa) -| (bb);
    %頂点Cと底面の左右の端点を結ぶ。つまり母線を引く。
    \draw[thick,blue] (a) -- node[left]{\large 母線$l$} (c);
    \draw (c) -- (b);
\end{tikzpicture}
\caption{円錐}
      \end{minipage} &
	%%%%第二の図%%%%
      \begin{minipage}[t]{0.35\linewidth}
      \begin{tikzpicture}
\draw [thick] (2,-3/2) -- (0,0);
\draw [thick] (-3/2,-2) -- (0,0);
\draw [domain=-1.5:2,smooth,thick] plot (\x, -{sqrt((5/2)^2-(\x)^2)});
\fill [red!30,domain=-1.5:2] (0,0) -- (-3/2,-2) -- plot (\x, -{sqrt((5/2)^2-(\x)^2)}) --(2,-3/2) -- cycle; 
\node [above,red] at ($(0,0)!1.7cm!(1/4,-5/4)$) {\large 側面積};
\fill ($(0,0)!3.2cm!(1/4,-5/4)$) circle[radius=1pt] coordinate(A);
\draw (A) circle [radius=0.7];
\end{tikzpicture}
\caption{円錐の展開図と側面積}
      \end{minipage}
	%%%%%%%
    \end{tabular}
\end{figure}

\begin{leftbar}
\underline{おまけ}\\
円錐などの立体的な図形の表面積や体積は、厳密には「積分」を使って求める。これに関しては、高校数学で学習する。だけど、解説されてる動画もyoutubeなどにあるとは思うので、興味がある方は調べてみるとよいかも。
\end{leftbar}

\begin{tcolorbox}
受験に必要な、図形に関する情報をまだまだ加筆していきます!!
\end{tcolorbox}

\end{document}