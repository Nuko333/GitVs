\documentclass{jsarticle}
% ltjsarticle: lualatex 用の 日本語 documentclass
% 他のタイプセットエンジンを使ってビルドする場合は、 \documentclass[dvipdfmx]{jsarticle} などとする。
\usepackage{url}
\usepackage[margin=20truemm]{geometry}

\begin{document}
\parindent = 0pt
\begin{center}
{\LARGE 読む文献の一覧} 
\end{center}
\begin{flushright}
\today
\end{flushright}
\section{プログラム・整理}
Python、CustomTkinterでフォームの作成\ ⇒\ やることを起動時に確認する?\\
Latexで、単元ごとに要点や定理まとめ\\
院試の文書作成\\
藤岡敦のHPのpdf
\section{微積分&線形代数}
黒田、「微分・積分」\\
齋藤毅、「線形代数の世界」\\
笠原、「微分積分学」\\
藤岡敦、「手を動かして学ぶ 続線形代数」\\
野本・岸、「解析演習」
\section{複素解析}
神保、「複素関数入門」\\
堀川、「複素関数の要諦」
\section{位相、多様体}
松坂、「集合・位相入門」\\
齋藤毅、「集合と位相」\\
松本、「多様体の基礎」
\section{微分幾何}
佐古彰史、「ゲージ理論・一般相対性理論のための微分幾何入門」\\
中内伸光、「じっくり学ぶ曲線と曲面―微分幾何学初歩」
\section{微分方程式}
笠原、「微分方程式の基礎」\\
ポントリャーギン、「常微分方程式」
\section{代数幾何、代数位相幾何}
宮西、「代数幾何」\\
桂、「代数幾何入門」\\
加藤、「位相幾何」\\
堀田、「代数入門」\\
星、「群論序説」\\
寺杣、「リーマン面の理論」\\
今野、「リーマン面と代数曲線」\\
アティマク\\
ハーツホーン、シルバーマン\\
Torsten wedhornの本\\
新妻、「可換環論の様相」\\
松村、「可換環論」
\section{ホモロジー代数、圏論}
志甫、ホモロジー代数\\
Goldblatt、『Topoi』
\section{院試}
大学院への演習シリーズ(代数、幾何、解析)\\
梶原、「新修解析学」\\
過去問を繰り返し解く☆☆
\section{webの文献}
Takatani Note\\
\url{https://takataninote.com/#%E5%BE%AE%E5%88%86%E7%A9%8D%E5%88%86}

\newpage
\section{院試の作成文章}
\subsection{志望分野}
$(1)$\\
$(2)$\\
$(3)$

\subsection{希望の教員}
$(1)$\\
$(2)$\\
$(3)$

\subsection{調査書の作成}
\fbox{1}\ 今までに読んだ数理科学関係の専門書で、第1志望分野に関するものを挙げる。著者とタイトルを書く(複数可)。一部分しか読んでいない場合も、どの部分を読んだか書いてください。\\ \newline
\fbox{2}\ その他に読んだ数理科学関係の専門書を書く。志望分野に選んだ分野以外のものも含めて、同様に挙げる。\\ \newline
\fbox{3}\ 特に面白いと感じた講義や関心をもった数理科学関連のトピックス。簡潔に書く(複数可)。\\ \newline
\fbox{4}\ 数理解析研究所で勉強したいと思う具体的テーマ(複数可)。現時点での希望で結構です。\newline

\subsection{テーマを一つ選んで、レポート作成}
テーマは次から一つ選ぶ。
\begin{itemize}
\item これまで学んだ数理科学関係の理論・定理で印象に残ったもの
\item 自分で取り組んだ問題
\item 自作のコンピュータ・プログラムなど
\end{itemize}
用紙のサイズは原則として A4 サイズとし、各ページは左右それぞれ 20mm の余白を設けてください。\\
長さは 1 ページ以上とし、長い場合には 1 ページのレジュメをつけてください。(理論・定理や問題の名称・概要を書き、その他感心した点、議論のポイント等をつけてください。)
\end{document}

